%%%%%%%%%%%%%%%%%%%%%%% file template.tex %%%%%%%%%%%%%%%%%%%%%%%%%
%
% This is a general template file for the LaTeX package SVJour3
% for Springer journals.          Springer Heidelberg 2010/09/16
%
% Copy it to a new file with a new name and use it as the basis
% for your article. Delete % signs as needed.
%
% This template includes a few options for different layouts and
% content for various journals. Please consult a previous issue of
% your journal as needed.
%
%%%%%%%%%%%%%%%%%%%%%%%%%%%%%%%%%%%%%%%%%%%%%%%%%%%%%%%%%%%%%%%%%%%
%
% First comes an example EPS file -- just ignore it and
% proceed on the \documentclass line
% your LaTeX will extract the file if required
%\begin{filecontents*}{example.eps}
%!PS-Adobe-3.0 EPSF-3.0
%%BoundingBox: 19 19 221 221
%%CreationDate: Mon Sep 29 1997
%%Creator: programmed by hand (JK)
%%EndComments
%gsave
%newpath
%  20 20 moveto
%  20 220 lineto
% 220 220 lineto
%  220 20 lineto
%closepath
%2 setlinewidth
%gsave
%  .4 setgray fill
%grestore
%stroke
%grestore
%\end{filecontents*}
%
%\RequirePackage{fix-cm}
%
%\documentclass{svjour3}                     % onecolumn (standard format)
\documentclass[smallcondensed]{svjour3}     % onecolumn (ditto)
%% Text Formatting
%% Manuscripts should be submitted in LaTeX. Please use Springer�s 
%% LaTeX macro package and choose the formatting option �smallcondensed�.
%\documentclass[smallextended]{svjour3}       % onecolumn (second format)
%\documentclass[twocolumn]{svjour3}          % twocolumn
%
\smartqed  % flush right qed marks, e.g. at end of proof
%
\usepackage{graphicx}
%
% \usepackage{mathptmx}      % use Times fonts if available on your TeX system
%
% insert here the call for the packages your document requires
%\usepackage{latexsym}
% etc.
\usepackage{amsmath}
\usepackage{amsfonts}
\usepackage{amssymb}
\usepackage{float}
\usepackage{color}
\usepackage{algorithm}
\usepackage{algorithmic}
\renewcommand{\algorithmicrequire}{\textbf{Input:}}
\renewcommand{\algorithmicensure}{\textbf{Output:}}
%
% please place your own definitions here and don't use \def but
% \newcommand{}{}
%
% Insert the name of "your journal" with
%\journalname{Numerical Algorithms} %Adv Comput Math
\journalname{Springer International Series of Numerical Mathematics??}
%
\begin{document}

\title{Balanced truncation for parametric linear systems
	using interpolation of gramians: a comparison of
	linear algebraic and geometric approaches}
%\subtitle{Application for a research assistant position at ICTEAM, UC Louvain}
%\thanks{}
%\thanks{Grants or other notes
%about the article that should go on the front page should be
%placed here. General acknowledgments should be placed at the end of the article.}
%\subtitle{Do you have a subtitle?\\ If so, write it here}

\titlerunning{Algebraic and Geometric interpolation of gramians for PMOR}        % if too long for running head

\author{SGMSA}  %etc.}

%\authorrunning{Short form of author list} % if too long for running head

%\institute{N. T. Son \at
%             INMA, ICTEAM, Universit/'e catholique de Louvain, 
%             Avenue Georges Lemaître 4-6/L4.05.01,\\
%             1348 Louvain-la-Neuve
%             \\
%              Tel: +32 10 47 80 10 \\
%             \email{thanh.son.nguyen@uclouvain.be}  \\%  if needed
%}

%\date{Received: date / Accepted: date}
% The correct dates will be entered by the editor


\maketitle

\begin{abstract}
In balanced truncation model order reduction, one has to solve a pair of Lyapunov equations for the two gramians and uses them for constructing a reduced-order
model. Although advances in solving such equations have been made, it is still the most
expensive step in this reduction method. For systems that depend on parameters, parametric model order reduction has to deal with the dependence on  parameters simultaneously with approximation of the input-output behavior of the full-order system. The
use of interpolation in parametric model order reduction has become popular. Nevertheless, interpolation of gramians is rarely mentioned, most probably due to the restriction to symmetric
positive semi-definite matrices. In this talk, we will present two approaches for interpolating
these structured matrices which are based on linear algebra and a recently developed Riemannian geometry. The result is then utilized in constructing parametric reduced-order
systems. Their numerical performances are compared on different models
\keywords{
Interpolation \and Parametric model order reduction \and Balanced truncation \and interpolation \and gramians \and Riemannian matrix manifold \and Symmetric positive semi-definite matrices of fixed rank 
}
% \PACS{PACS code1 \and PACS code2 \and more}
\subclass{
%15A22    % Matrix pencils 
%\and
 65D05 % Interpolation
\and 65F30 % Other matrix algorithms
\and 93C05 % Control systems, Linear systems
%\and 94C % Circuits, networks
\and notdoneyet
}
\end{abstract}


\section{Introduction}\label{Sec:Intro}
I cite somebody here \cite{AbsiMS08}.
\section{Brief balanced truncation for parametric linear systems and standard interpolation}\label{Sec:BT_standard interpolation}
\subsection{Balanced truncation}
\subsection{Interpolation of gramians for parametric model order reduction}

\section{Manifold $\mathcal{S}_+(k,n)$ and its interpolation scheme}\label{Sec:Manifold}
\subsection{A quotient geometry of $\mathcal{S}_+(k,n)$}
\subsection{Curve and surface interpolation for parametric model order reduction}
\subsection{A note on imbedded geometry of $\mathcal{S}_+(k,n)$}

\section{Numerical examples}\label{Sec:NumerExam}


\section{Conclusion}\label{Sec:Concl}




% BibTeX users please use one of
%\bibliographystyle{spbasic}      % basic style, author-year citations
\bibliographystyle{spmpsci}      % mathematics and physical sciences
%\bibliographystyle{spphys}       % APS-like style for physics
\bibliography{references}   % name your BibTeX data base

% Non-BibTeX users please use
%\begin{thebibliography}{}
%
% and use \bibitem to create references. Consult the Instructions
% for authors for reference list style.
%
%\bibitem{RefJ}
%% Format for Journal Reference
%Author, Article title, Journal, Volume, page numbers (year)
%% Format for books
%\bibitem{RefB}
%Author, Book title, page numbers. Publisher, place (year)
%% etc

\end{document}
% end of file template.tex

